\documentclass{article}
\newcommand\tab[1][1cm]{\hspace*{#1}}
\usepackage[margin=60pt]{geometry}
\usepackage[utf8]{inputenc}
\usepackage{hyperref}
\usepackage{makeidx}\usepackage{graphicx}
\makeindex
\begin{document}
\begin{titlepage}
   \vspace*{\stretch{1.0}}
   \begin{center}
      \Huge\textbf{Home Challenge \#1: Sniffing}\\
      \vspace{5mm} %5mm vertical space
      \Large Prof. Matteo Cesana - Year 2020/2021\\
      \vspace{5mm} %5mm vertical space
      \large\textit{Gabba Rohit [codice persona: 10706944]}
      \linebreak
      \large\textit{Tortorelli Giuseppe [codice persona: 10582962]}
   \end{center}
   \vspace*{\stretch{2.0}}
\end{titlepage}
\printindex

%\tableofcontents
%\pagebreak

%1
\section{What’s the difference between the message with MID:
3978 and the one with MID: 22636?}
Using Wireshark filter: \[ coap.mid == 3978 \; \&\&  \; coap.mid == 22636 \] we have isolated the two packets. After that we noticed that they have a different token and also that the first is of type \textit{CON} and the second of type \textit{NON}. Therefore the first one requires an \textit{ACK} message and the second one does not.

%2
\section{Does the client receive the response of message No.
6949?}
First of all using Wireshark \textit{No.} column we found the message number 6949 and we read its \textit{MID}. After this using filter: \[ coap.mid == 28357 \] we discovered that there was an \textit{ACK} containing the error message \textit{4.05}, which means the method is not allowed therefore the client did not receive any other response.

%3
\section{How many replies of type confirmable and result code
“Content” are received by the server “localhost”?}
Using filter: \[ coap.type == 0 \; \&\&  \; ip.dst == 127.0.0.1 \; \&\& \; coap.code == 69 \] we found out that there were 8 replies.

%4
\section{How many messages containing the topic
“factory/department*/+” are published by a client with
user name: “jane”? Where * replaces the dep. number,
e.g. factory/department1/+, factory/department2/+
and so on?}
Using filter: \[ mqtt.username == jane\] we got 4 connection messages. Abbiamo seguito il flusso TCP di questi 4 messaggi trovando 4 diverse porte (50985, 40989, 42821, 40004). \hfill \break
Usando il filtro: \[ mqtt.topic \; contains \; "factory" \; \&\& \; ( tcp.port == 50985 \; || \; tcp.port == 40989 \; || \; tcp.port == 42821 \; || \; tcp.port == 40004 ) \; \&\& \; mqtt.msgtype == 3 \] 
abbiamo trovato 10 messaggi ma non esiste alcun messaggio del livello richiesto poiché essendoci nella richiesta un + e non un \# , i publish messages sono tutti di livello inferiore.

%5
\section{How many clients connected to the broker “hivemq”
have specified a will message?}
Using filter: \[ ip.dst\_host == broker.hivemq.com \; \&\& \; mqtt.msgtype == 1 \; \&\& \; mqtt.willmsg\_len > 0 \] we have selected
all connection messages with \textit{will message} specified sent to \textit{hivemq}. Counting also the connection messages with the same \textit{ClientId} and those in which the \textit{CleintId} was not specified we found 16 messages
dei quali 3 hanno ClinetId distinti, uno è duplicato, il resto non ha specificato ClientId

%6
\section{How many publishes with QoS 1 don't receive the ACK?}
Using filter: \[ mqtt.msgtype == 3 \; \&\& \; mqtt.qos == 1 \] we got all published messages with QoS equals to 1 (124). \hfill \break
Using filter: \[ mqtt.msgtype == 4 \] we got all \textit{ACK} messages for the published one (74). \hfill \break
By computing the difference \[ 124-74 = 50\] we obtained the number of messages that did not receive an \textit{ACK}.
Abbiamo considerato anche eventuali publish messages duplicati

%7
\section{How many last will messages with QoS set to 0 are
actually delivered?}
Abbiamo interpretando la domanda come tutti i last will messages ricevuti, poiché non c'è modo di sapere i last will messages specificati da i clienti della cattura verso terzi. Quindi abbiamo per prima cosa notato che tutti i will messages specificati nei messaggi di connect, contenevano 
la parola "error".Usando il filtro \[ mqtt.msgtype == 3 \; \&\& \; mqtt.qos == 0 \; \&\& \; mqtt.msg contains "error" \]
per avere tutti i messaggi pubblicati contenenti la parola error (dedotti essere last will messages ricevuti), abbiamo trovato un solo messaggio.

%8
\section{Are all the messages with QoS\textgreater  0 published by the
client \break“4m3DWYzWr40pce6OaBQAfk” correctly delivered
to the subscribers?}
Using filter: \[ mqtt.clientid == "4m3DWYzWr40pce6OaBQAfk"\] we found sender ip address (10.0.2.15), source port (58313), destination ip (5.196.95.208). \hfill \break
Using filter: \[ ip.src == 10.0.2.15 \; \&\& \; tcp.srcport == 58313 \; \&\& \; mqtt.qos > 0\; \&\& \; mqtt.msgtype == 3\] we found that the client sent one pub message with \textit{QoS} set to 2, to the same server as previous connection message. \hfill \break
Using filter: \[ ip.src == 5.196.95.208 \; \&\& \; tcp.dstport == 58313 \; \&\& \; ip.dst == 10.0.2.15 \; \&\& \; mqtt.msgtype == 5\] we got the \textit{PUBREC} message (same \textit{Message Identifier} equal to 3) sended from the broker to the client. \hfill \break
It means that the borker successfully received the message and has sent it to the \textit{"factory/department1/section1/deposit"} subscribers correctly. So it can be said that all publish messages sended by the client have been correctly delivered even if there is no a \textit{PUBREL} message sended to the server 


%9
\section{What is the average message length of a connect msg
using mqttv5 protocol? Why messages have different
size?}
Using filer: \[ mqtt.ver == 5\] we got all connection messages that use mqtt version 5. \hfill \break
Analyzing the \textit{"Msg Len"} field of each message we obtained: 
\begin{itemize} 
   \item 35 messages of length 13
   \item 1 message of length 29
   \item 4 messages of length 69
   \item 4 messages of length 65
   \item 1 message of length 20
   \item 4 messages of length 77
   \item 2 messages of length 32
   \item 5 messages of length 25
   \item 1 message of length 27
   \item 1 message of length 86
   \item 1 message of length 33
   \item 2 messages of length 30
   \item 1 message of length 78
   \item 1 message of length 83
\end{itemize}
Calculating the average, we obtained an average length of 30.22 bytes. \hfill \break
The length changes from message to message because the connection message may or may not contain additional information such as the last will message.
(Nella versione di wireshark usata per analizzare il pcapng non c'erano malformed messages)

%10
\section{Why there aren’t any REQ/RESP pings in the pcap?}
There aren’t any REQ/RESP pings in the pcap because the clients and the broker keep interacting without ever
letting the Keep Alive timers of the clients expire.


\end{document}
