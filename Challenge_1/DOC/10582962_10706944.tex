\documentclass{article}
\newcommand\tab[1][1cm]{\hspace*{#1}}
\usepackage[margin=60pt]{geometry}
\usepackage[utf8]{inputenc}
\usepackage{hyperref}
\usepackage{makeidx}\usepackage{graphicx}
\makeindex
\begin{document}
\begin{titlepage}
   \vspace*{\stretch{1.0}}
   \begin{center}
      \Huge\textbf{Home Challenge \#1: Sniffing}\\
      \vspace{5mm} %5mm vertical space
      \Large Prof. Matteo Cesana - Year 2020/2021\\
      \vspace{5mm} %5mm vertical space
      \large\textit{Gabba Rohit [codice persona: 10706944]}
      \linebreak
      \large\textit{Tortorelli Giuseppe [codice persona: 10582962]}
   \end{center}
   \vspace*{\stretch{2.0}}
\end{titlepage}
\printindex

%\tableofcontents
%\pagebreak

%1
\section{What’s the difference between the message with MID:
3978 and the one with MID: 22636?}
Using Wireshark filter: \[ coap.mid == 3978 \; \&\&  \; coap.mid == 22636 \] we have isolated the two packets. After that we noticed that they have a different token and also that the first is of type \textit{CON} and the second of type \textit{NON}. Therefore the first one requires an \textit{ACK} message and the second one does not.

%2
\section{Does the client receive the response of message No.
6949?}
First of all using Wireshark \textit{No.} column we found the message number 6949 and we read its \textit{MID}. After this using filter: \[ coap.mid == 28357 \] we discovered that there was an \textit{ACK} containing the error message \textit{4.05}, which means the method is not allowed therefore the client did not receive any other response.

%3
\section{How many replies of type confirmable and result code
“Content” are received by the server “localhost”?}
Using filter: \[ coap.type == 0 \; \&\&  \; ip.dst == 127.0.0.1 \; \&\& \; coap.code == 69 \] we found out that there were 8 replies.

%4
\section{How many messages containing the topic
“factory/department*/+” are published by a client with
user name: “jane”? Where * replaces the dep. number,
e.g. factory/department1/+, factory/department2/+
and so on?}

%5
\section{How many clients connected to the broker “hivemq”
have specified a will message?}
First of all we found "hivemq" ip addresses through \textit{DNS} messages (18.185.199.22, 3.120.68.56). Using filter: \[ mqtt.willmsg \; \&\& \; mqtt.clientid\_len > 0 \] we have selected
all connection messages with \textit{will flag} set and sent by not null \textit{ClientID}. Browsing all the entries we counted 3 different clients.

%6
\section{How many publishes with QoS 1 don't receive the ACK?}
Using filter: \[ mqtt.msgtype == 3 \; \&\& \; mqtt.qos == 1 \] we got all published messages with QoS equals to 1 (124). \hfill \break
Using filter: \[ mqtt.msgtype == 4 \] we got all \textit{ACK} messages for the published one (74). \hfill \break
By computing the difference \[ 124-74 = 50\] we obtained the number of messages that did not receive an \textit{ACK}.

%7
\section{How many last will messages with QoS set to 0 are
actually delivered?}

%8
\section{Are all the messages with QoS\textgreater  0 published by the
client \break“4m3DWYzWr40pce6OaBQAfk” correctly delivered
to the subscribers?}
Using filter: \[ mqtt.clientid == "4m3DWYzWr40pce6OaBQAfk"\] we found sender ip address (10.0.2.15), source port (58313), destination ip (5.196.95.208). \hfill \break
Using filter: \[ ip.src == 10.0.2.15 \; \&\& \; tcp.srcport == 58313 \; \&\& \; mqtt.qos > 0\; \&\& \; mqtt.msgtype == 3\] we found that the client sent one pub message with \textit{QoS} set to 2. \hfill \break
Using filter: \[ ip.src == 5.196.95.208 \; \&\& \; tcp.dstport == 58313 \; \&\& \; ip.dst == 10.0.2.15 \; \&\& \; mqtt.msgtype == 5\] we got the \textit{PUBREC} message sended from the broker to the client. \hfill \break
Not having found any \textit{PUBREL} message, it means that the borker successfully received the message but did not send it to any other client. So no messages published by “4m3DWYzWr40pce6OaBQAfk” is correctly delivered.


%9
\section{What is the average message length of a connect msg
using mqttv5 protocol? Why messages have different
size?}
Using filer: \[ mqtt.ver == 5\] we got all connection messages that use mqtt version 5. \hfill \break
Analyzing them we obtained: 
\begin{itemize} 
   \item 35 messages of length 13
   \item 1 message of length 29
   \item 4 messages of length 69
   \item 4 messages of length 65
   \item 1 message of length 20
   \item 4 messages of length 77
   \item 2 messages of length 32
   \item 5 messages of length 25
   \item 1 message of length 27
   \item 1 message of length 86
   \item 1 message of length 33
   \item 2 messages of length 30
   \item 1 message of length 78
   \item 1 message of length 83
\end{itemize}
Calculating the average, we obtained an average length of 30.22 bytes. \hfill \break
The length changes from message to message because the connection message may or may not contain additional information such as the last will message.

%10
\section{Why there aren’t any REQ/RESP pings in the pcap?}
There aren't any REQ/RESP pings in the pcap because the clients and the broker keep ingeracting without ever letting the Keep Alive timers of the clients expire.

\end{document}
