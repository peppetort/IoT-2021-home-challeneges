  
\documentclass{article}
\newcommand\tab[1][1cm]{\hspace*{#1}}
\usepackage[margin=60pt]{geometry}
\usepackage[utf8]{inputenc}
\usepackage{hyperref}
\usepackage{makeidx}\usepackage{graphicx}
\makeindex
\begin{document}
\begin{titlepage}
   \vspace*{\stretch{1.0}}
   \begin{center}
      \Huge\textbf{Home Challenge \#4: TinyOS 2}\\
      \vspace{5mm} %5mm vertical space
      \Large Prof. Matteo Cesana - Year 2020/2021\\
      \vspace{5mm} %5mm vertical space
      \large\textit{Gabba Rohit [codice persona: 10706944]}
      \linebreak
      \large\textit{Tortorelli Giuseppe [codice persona: 10582962]}
      \linebreak
      \linebreak
      \linebreak
      \linebreak
   \end{center}
   \vspace*{\stretch{2.0}}
\end{titlepage}

\pagebreak

\section{SendAck.h}
In this file we described the message struct, composed of three field:

\begin{itemize}
	\item \textit{msg\_type}: which describes the type of message (REQ/RESP).
	\item \textit{msg\_counter}: which keeps track of the message number.
	\item \textit{msg\_value}: which represents the value read from the fake sensor.
\end{itemize}

\section{SendAckAppC.nc}
We used this file to wire all the modules defined in the \textit{SendAckC.nc} file. And we also wired the \textit{fakeSensorC.nc} file to read the values from the fake sensor.


\section{SendAckC.nc}
In this file we implemented all the logic of the single modules. We intensively used the DEBUG statements to make the debugging process easier. \hfill \break
We used the variable TOS\_NODE\_ID to differentiate the functions each mote has to run and used the module \textit{PacketAcknowledgements} to send the ACK messages.

\section{topology.txt}
We used this file to describe the topology of our motes.

\section{RunSimulationScript.py}
This python script is used to initialize the motes and run the simulation. In the end all the log is printed in a text file.



\end{document}
