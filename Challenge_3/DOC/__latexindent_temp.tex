\documentclass{article}
\newcommand\tab[1][1cm]{\hspace*{#1}}
\usepackage[margin=60pt]{geometry}
\usepackage[utf8]{inputenc}
\usepackage{hyperref}
\usepackage{makeidx}\usepackage{graphicx}
\makeindex
\begin{document}
\begin{titlepage}
   \vspace*{\stretch{1.0}}
   \begin{center}
      \Huge\textbf{Home Challenge \#3: TinyOS}\\
      \vspace{5mm} %5mm vertical space
      \Large Prof. Matteo Cesana - Year 2020/2021\\
      \vspace{5mm} %5mm vertical space
      \large\textit{Gabba Rohit [codice persona: 10706944]}
      \linebreak
      \large\textit{Tortorelli Giuseppe [codice persona: 10582962]}
      \linebreak
      \linebreak
      \linebreak
      \linebreak
   \end{center}
   \vspace*{\stretch{2.0}}
\end{titlepage}

\pagebreak

\section{Tiny-OS Code}
Abbiano iniziato creando il file \textit{foo.h} in cui abbiamo specificato la struttura dei messaggi.
Abbiamo creato un campo \textit{id} per indicare l'id del sender e un campo \textit{counter} per il contatore.
Abbiamo quindi creato il file \textit{fooC.nc} in cui abbiamo implementato la logica.
Abbiamo creato le variabili:
\begin{itemize}
   \item \textit{bool lock}: per controllare che i messaggi inviati non si sovrappongano
   \item \textit{uint16\_t counter}: per rappresentare il contatore locale di ogni mote (inizializzato a 0)
   \item \textit{bool led0, led1, le2}: per rappresentare lo stato dei led di ogni mote
\end{itemize}
Abbiamo impostato la frequenza di ogni mote come indicato da specifiche e implementato le funzioni di send con messaggi in broadcast e receive.


\section{Cooja Simulation}
Creata una nuova simulazione abbiamo impostato 3 nuovi sky motes compilati tramite \textit{make telosb}.
Tramite delle \textit{printf} di log abbiamo controllato che i messaggi venissero inviati e ricevuti correttamente.
Abbiamo quindi stampato i valori richiesti per il mote 2 in ordine (\textit{led2 led1 led0}).


\end{document}
