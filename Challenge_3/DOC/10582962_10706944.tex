\documentclass{article}
\newcommand\tab[1][1cm]{\hspace*{#1}}
\usepackage[margin=60pt]{geometry}
\usepackage[utf8]{inputenc}
\usepackage{hyperref}
\usepackage{makeidx}\usepackage{graphicx}
\makeindex
\begin{document}
\begin{titlepage}
   \vspace*{\stretch{1.0}}
   \begin{center}
      \Huge\textbf{Home Challenge \#3: TinyOS}\\
      \vspace{5mm} %5mm vertical space
      \Large Prof. Matteo Cesana - Year 2020/2021\\
      \vspace{5mm} %5mm vertical space
      \large\textit{Gabba Rohit [codice persona: 10706944]}
      \linebreak
      \large\textit{Tortorelli Giuseppe [codice persona: 10582962]}
      \linebreak
      \linebreak
      \linebreak
      \linebreak
   \end{center}
   \vspace*{\stretch{2.0}}
\end{titlepage}

\pagebreak

\section{Tiny-OS Code}
We created the file \textit{foo.h} which conatains the structure of the messages, composed by the variable \textit{id} (which represents the sender id) and the variable \textit{counter} to count the number of messages received.
We then created the file \textit{fooC.nc} which contains the model of the application.
We defined some variables:
\begin{itemize}
   \item \textit{bool lock}: to make sure that the same sender does not send multiple messages at the same time..
   \item \textit{uint16\_t counter}: to represent the local counter of each mote (initialized with 0).
   \item \textit{bool led0, led1, le2}: to represent the mote's LEDs state.
\end{itemize}
We configured the frequency of each mote as specified and implemented the functions to send and receive the messages.

\section{Cooja Simulation}
We started a new simulation by creating 3 new sky motes using the main.exe file compiled using the \textit{make telosb} command.
Using the \textit{printf} functions we made sure that the messages were sent and received properly.
We, finally, printed the first 20 values received by the mode \#2 in order (\textit{led2 led1 led0}): 000,100,101,001,101,001,101,001,000,100,\linebreak000,100,000,100,101,000,100,000,100,000.


\end{document}
