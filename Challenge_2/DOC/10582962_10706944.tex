\documentclass{article}
\newcommand\tab[1][1cm]{\hspace*{#1}}
\usepackage[margin=60pt]{geometry}
\usepackage[utf8]{inputenc}
\usepackage{hyperref}
\usepackage{makeidx}\usepackage{graphicx}
\makeindex
\begin{document}
\begin{titlepage}
   \vspace*{\stretch{1.0}}
   \begin{center}
      \Huge\textbf{Home Challenge \#2: Node-RED}\\
      \vspace{5mm} %5mm vertical space
      \Large Prof. Matteo Cesana - Year 2020/2021\\
      \vspace{5mm} %5mm vertical space
      \large\textit{Gabba Rohit [codice persona: 10706944]}
      \linebreak
      \large\textit{Tortorelli Giuseppe [codice persona: 10582962]}
      \linebreak
      \linebreak
      \linebreak
      \href{https://www.thingspeak.com/channels/1359547}{https://www.thingspeak.com/channels/1359547}
      \linebreak
   \end{center}
   \vspace*{\stretch{2.0}}
\end{titlepage}

\pagebreak
\section{Node-RED}
Abbiamo fatto il parsing del CSV fornito usando come separatore la virgola. Per una migliore leggibilità del codice abbiamo impostato il nodo \textit{csv}
in modo tale da restituire tutti i valori in un singolo array. \hfill \break 

Per ogni valore dell'array (corrispondente ad una riga del csv) abbaiamo ispezionato tutte le colonne e abbiamo filtrato le righe contenenti un 
\textit{Publish Message}. \hfill \break
A questo punto abbiano notato che il \textit{Publish Message} associato ad ogni \textit{payload} era ripetuto più volte. Quindi abbiamo creato un algoritmo 
per mappare ogni \textit{payload} con il corrispondente \textit{Publish Message} (considerando che erano ripetuti secondo un ordine sequenziale). \hfill \break 

Dai messaggi ottenuti abbiamo quindi proseguito creando due diramazioni, una per filtrare tutti i messaggi con topic \textit{factory/department1/section1/plc} oppure 
\textit{factory/department3/section3/plc} e l'altra per tutti i messaggi aventi come topic \textit{factory/department1/section1/hydraulic\_valve} oppure 
\textit{factory/department3/section3/hydraulic\_valve}. \hfill \break 
Dopo aver decodificato i messaggi ed estratto il valore richeisto, abbiamo aggiunto un nodo \textit{split} per procedere all'invio singolo dei valori.
\hfill \break I messaggi sono stati inviati con un intervallo di 30 secondi tra uno e l'altro.

\section{ThingSpeak}
Creato un nuovo canale, abbiamo aggiunto due \textit{fields} una per i valori di \textit{plc} ed una per quelli di \textit{hydraulic\_valve} \hfill \break
Abbiamo aggiunto anche due bottoni ognuno associatp ad una \textit{field} impostato il valore di accensione dai 2000 in su.

\hfill \break 
Link a \textit{ThingSpeak}:  \href{https://www.thingspeak.com/channels/1359547}{https://www.thingspeak.com/channels/1359547}

\section{Conclusione}
Abbiamo trovato 20 valori riguardanti \textit{plc}: 4,403,21,66,66,66,764,32,32,36,36,5,66,1747,4,31,764,14,2010,1380  \hfill \break 
e 29 valori riguardanti \textit{hydraulic\_valve}: 2,1344,14,638,14,1344,60,11,559,30,42,20,3162,14,195,2,14,14,14,3162,3162,1,1,39,14,1344,1344,3162,39



\end{document}
