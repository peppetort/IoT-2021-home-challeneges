\documentclass{article}
\newcommand\tab[1][1cm]{\hspace*{#1}}
\usepackage[margin=60pt]{geometry}
\usepackage[utf8]{inputenc}
\usepackage{hyperref}
\usepackage{makeidx}\usepackage{graphicx}
\makeindex
\begin{document}
\begin{titlepage}
   \vspace*{\stretch{1.0}}
   \begin{center}
      \Huge\textbf{Home Challenge \#2: Node-RED}\\
      \vspace{5mm} %5mm vertical space
      \Large Prof. Matteo Cesana - Year 2020/2021\\
      \vspace{5mm} %5mm vertical space
      \large\textit{Gabba Rohit [codice persona: 10706944]}
      \linebreak
      \large\textit{Tortorelli Giuseppe [codice persona: 10582962]}
      \linebreak
      \linebreak
      \linebreak
      \href{https://www.thingspeak.com/channels/1359547}{https://www.thingspeak.com/channels/1359547}
      \linebreak
   \end{center}
   \vspace*{\stretch{2.0}}
\end{titlepage}

\pagebreak
\section{Node-RED}
Abbiamo fatto il parsing del CSV fornito usando come separatore la virgola. Dal momento che il file parsato restituiva alcuni valori di difficile interpretazione,
abbiamo deciso di ispezionare tutte le colonne di ogni riga. \hfill \break \break
 Quindi abbiamo filtrato il contenuto selezionando le celle non nulle contenenti un \textit{Publish Message}. \hfill \break 
Per ogni cella selezioanata abbiamo filtrato anche tutti i messaggi contenenti il carattere esadecimale \textit{7b} ('\{')
così da essere sicuri di avere messaggi aventi un reale contenuto associato.\hfill \break \break
Dai messaggi ottenuti abbiamo quindi proseguito creando due diramazioni, una per filtrare tutti i messaggi con topic \textit{factory/department1/section1/plc} oppure 
\textit{factory/department3/section3/plc} e l'altra per tutti i messaggi aventi come topic \textit{factory/department1/section1/hydraulic\_valve} oppure 
\textit{factory/department3/section3/hydraulic\_valve}. \hfill \break
Dopodiché abbiamo isolato il messaggio utilizzando gli estremi \textit{7b} ('\{') e \textit{7d} ('\}') e quindi decodificato. \hfill \break
Usando il formato \textit{JSON} abbiamo estratto il campo '\textit{value}' ed impostato il \textit{msg.topic} e \textit{msg.paylod} per iniviare il valore tramite mqtt. \hfill \break \break
Per permetter a \textit{ThingSpeak} di ricevere correttamente tutti i valori, abbiamo impostato un l'invio di un messaggio ogni 30 secondi im maniera alternata tra i valori di \textit{plc} e quelli di  \textit{hydraulic\_valve}

\section{ThingSpeak}
Creato un nuovo canale, abbiamo aggiunto due \textit{fields} una per i valori di \textit{plc} ed una per quelli di \textit{hydraulic\_valve} \hfill \break
Abbiamo aggiunto anche due bottoni ognuno associatp ad una \textit{field} impostato il valore di accensione dai 2000 in su.

\hfill \break 
Link a \textit{ThingSpeak}:  \href{https://www.thingspeak.com/channels/1359547}{https://www.thingspeak.com/channels/1359547}

\section{Conclusione}
Abbiamo trovato 19 valori riguardanti \textit{plc}: 4,21,66,66,66,764,32,32,36,36,5,66,1747,4,31,764,14,2010,1380 \hfill \break 
e 17 valori riguardanti \textit{hydraulic\_valve}: 2,1344,2501,60,11,559,30,42,20,3162,195,1,1,39,1344,1344,39



\end{document}
