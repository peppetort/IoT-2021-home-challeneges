\documentclass{article}
\newcommand\tab[1][1cm]{\hspace*{#1}}
\usepackage[margin=60pt]{geometry}
\usepackage[utf8]{inputenc}
\usepackage{hyperref}
\usepackage{makeidx}\usepackage{graphicx}
\makeindex
\begin{document}
\begin{titlepage}
   \vspace*{\stretch{1.0}}
   \begin{center}
      \Huge\textbf{Home Challenge \#2: Node-RED}\\
      \vspace{5mm} %5mm vertical space
      \Large Prof. Matteo Cesana - Year 2020/2021\\
      \vspace{5mm} %5mm vertical space
      \large\textit{Gabba Rohit [codice persona: 10706944]}
      \linebreak
      \large\textit{Tortorelli Giuseppe [codice persona: 10582962]}
      \linebreak
      \linebreak
      \linebreak
      \href{https://www.thingspeak.com/channels/1359547}{https://www.thingspeak.com/channels/1359547}
      \linebreak
   \end{center}
   \vspace*{\stretch{2.0}}
\end{titlepage}

\pagebreak
\section{Node-RED}
We parsed the provided CSV file using the comma separator. To make the code more readable we made the \textit{csv} node return all the file content in a single array.\hfill \break 

For each value of the array (which correspond to the lines of the csv file) we examined every single column, keeping only the lines that contain at least one \textit{Publish Message}. \hfill \break

We then noticed that \textit{Publish Message} associated to each \textit{payload} was replicated several times. Therefore we created an algorithm to map each \textit{payload} with its corresponding \textit{Publish Message} (exploiting the fact that the replicas of the \textit{Publish Messages} were in a sequential order). \hfill \break 

Filtering out all the messages according to the topics we created two branches: one for all the messages of topics \textit{factory/department1/section1/plc} or \textit{factory/department3/section3/plc} and the other one for the messages having the topics \textit{factory/department1/section1/hydraulic\_valve} or \textit{factory/department3/section3/hydraulic\_valve}. \hfill \break 

In the end, after decoding the messages and extracting the required values, we used the \textit{split} node to send the messages one-by-one through the \textit{MQTT} protcol.\hfill \break 

The messages have been sent with an interval of 30 seconds between two consecutive ones.

\section{ThingSpeak}
We created a new channel with two \textit{fields}: one for the values of topic \textit{plc} and the other one for those of the topic \textit{hydraulic\_valve}. \hfill \break
We also created two lamp indicators, one for each field, which turn ON when the value of the \textit{field} is equal to or greater than 2000.

\hfill \break 
Link of the \textit{ThingSpeak} channel:  \href{https://www.thingspeak.com/channels/1359547}{https://www.thingspeak.com/channels/1359547}

\section{Conclusione}
20 values of topic \textit{plc}: \hfill\break
4,403,21,66,66,66,764,32,32,36,36,5,66,1747,4,31,764,14,2010,1380 \hfill \break \break
29 values of the topic \textit{hydraulic\_valve}: \hfill \break 2,1344,14,638,14,1344,60,11,559,30,42,20,3162,14,195,2,14,14,14,3162,3162,1,1,39,14,1344,1344,3162,39



\end{document}
