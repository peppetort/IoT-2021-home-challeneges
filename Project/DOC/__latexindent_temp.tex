\documentclass{article}
\newcommand\tab[1][1cm]{\hspace*{#1}}
\usepackage[margin=60pt]{geometry}
\usepackage[utf8]{inputenc}
\usepackage{hyperref}
\usepackage{makeidx}
\usepackage{graphicx}
\graphicspath{{./res/}}
\makeindex
\begin{document}
\begin{titlepage}
   \vspace*{\stretch{1.0}}
   \begin{center}
      \Huge\textbf{IoT Project: "Keep your distance"}\\
      \vspace{5mm} %5mm vertical space
      \Large Prof. Matteo Cesana - Year 2020/2021\\
      \vspace{5mm} %5mm vertical space
      \large\textit{Gabba Rohit [codice persona: 10706944]}
      \\
      \large\textit{Tortorelli Giuseppe [codice persona: 10582962]}
   \end{center}
   \vspace*{\stretch{2.0}}
\end{titlepage}
\printindex

\section{TinyOS}
Al fine di identificare i mesaggi consecutivi abbiamo scelto di 
associare ad ogni messaggio un id incrementale. \\
La struttura è la seguente:

\begin{itemize}
   \item \textit{id}: contentente l'id di ogni mote
   \item \textit{msg\_number}: l'id del messaggio
\end{itemize}

Abbiamo scelto di associare ad ogni mote 3 memorie di grandezza fissata:

\begin{itemize}
   \item \textit{near\_motes\_ids}: per salvare l'id dei nodi vicini 
   \item \textit{near\_motes\_counters}: per contare quanti messaggi consecutivi sono stati ricevuti 
   \item \textit{near\_motes\_msgs\_number}: per salvare l'id del messaggio che ci si aspetta di ricevere
\end{itemize}

Ogni volta che un mote invia un messaggio, incrementa l'id da associare al messaggio e lo invia.\\\\
Alla ricezione di un messaggio, si scorre l'array \textit{near\_motes\_ids} in cerca dell'id del mote da cui è stato 
ricevuto il messaggio. \\
Se l'id non viene trovato, viene aggiunto nella prima cella disponibile.
Altrimenti si procede con la verifica dell'id del messaggio. \\
Quindi si legge il valore aspettato dall'array \textit{near\_motes\_msgs\_number} e si confronta con quello ricevuto. 
Se i due valori non corrispondo, il contatore associato al mote viene azzerato.\\\\
Si procede quindi alla verifica del valore contenuto nell'array \textit{near\_motes\_counters}: nel caso sono stati
ricevuti 10 messaggi consecutivi, si lancia l'allarme e si azzera il contatore, altrimenti il contatore viene incrementato.

\section{Cooja}
Abbiamo testato il funzionamento in \textit{Cooja} inserendo 5 \textit{Sky mote} e provando varie configurazioni di questi nella Network.\\
Nell'implementazioe abbiamo assunto che la ricezione dei messaggi avvenga senza interferenze, per questo abbiamo
ignorato i casi in cui il contatore viene resettato a causa di un messaggio non consegnato correttamente.
\\\\
Di seguito gli screenshots delle configurazioni testate (sono ordinati secondo il log file)
\\
\includegraphics[scale=0.1]{inserire_immagine_qui_e_in_cartella_res.png}
\\
Ad ogni nodo abbiamo associato un socket verso l'ip di \textit{Node-RED} ognuno con una porta differente (da 6001 a 6005)

\section{Node-RED}

Abbiamo creato 5 (numero di mote scelti per il testing) nodi \textit{tcp in} ognuno ascoltante sulla porta del nodo associato.\\
Abbiamo quindi inserito un nodo \textit{function} per filtrare i messaggi di allarme ed estrarne gli id dei mote interessati.\\
Infine abbiamo utilizzato un nodo \textit{htttp request} per mandare una POST request verso \textit{IFTTT} specificando la nostra API key

\section{IFTTT}
Ricevuti i dati tramite un applet \textit{Webhooks} ascoltante sull'evento \textit{ALARM},
inviamo una notifica sull'app \textit{IFTTT} per cellulare con il seguente testo:\\\\
\textit{"ALARM mote x: I'm close to mote y! COVID"}\\\\
Dove \textit{x} è l'id del mote che ha inviato il messaggio.

\end{document}
